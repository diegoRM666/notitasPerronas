\documentclass[12p]{report}
\usepackage{helvet}
\usepackage[left=3cm,right=3cm,top=2.5cm,bottom=2.5cm]{geometry}
\renewcommand{\familydefault}{\sfdefault}


% Title Page
\title{Ensayo: Huella e imágenes de las mujeres}
\author{Diego Ruiz Mora | 2202000335}


\begin{document}
\maketitle
\newpage
\subsection{Introducción}
Es importante denotar la huella y la imagen de la mujer en cada uno de los ámbitos que la rodean, es de suma importancia si queremos entender la situación a la que se ven sometidas desde el inicio de la civilización como lo conocemos. 
\\\\
Es fácil intuir que su papel no ha sido de lo más dignificante, observando que aun hoy en día existen rezagos del papel de estas en algún punto de la historia. Como lo podemos observar en nuestro día a día, y a pesar de la constante lucha por la igualdad,  veremos que el papel de las mujeres se ve reducido en todos los aspectos de la vida pública, basta con alejarnos un poco de las grandes ciudades para hacer más visible las diferencias entre el papel de los hombres y las mujeres tanto en las esferas políticas, sociales y laborales. 
\\\\
Para este caso en particular, abordaremos la perspectiva en un punto especifico en el tiempo, en concreto, la Edad Media, que desde mi muy personal punto de vista, es un época donde no existía un interés propio por ahondar en los temas de carácter social. 
Existía la esclavitud y la monarquía, dos puntos que contrastan muchísimo con la idea de igualdad para la sociedad, ya que no solo era poco alcanzable, si no que era prácticamente imposible salir de una esfera social para entrar a otra. 
\\\\
Dada situación no excluía a las mujeres, al contrario, las rodea con una tal intensidad que es complicado de  

\end{document}          
