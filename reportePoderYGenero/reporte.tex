\documentclass{report}
\usepackage{setspace}
\usepackage{titlesec}
\usepackage{helvet}
\usepackage[left=3cm,right=3cm,top=2.5cm,bottom=2.5cm]{geometry}

\renewcommand{\familydefault}{\sfdefault}
\titleformat*{\section}{\LARGE\bfseries}
\titleformat*{\subsection}{\Large\bfseries}
\titleformat*{\subsubsection}{\large\bfseries}
\titleformat*{\paragraph}{\large\bfseries}
\titleformat*{\subparagraph}{\large\bfseries}


% Title Page
\title{Ensayo: Huella e imágenes de las mujeres}
\author{Diego Ruiz Mora | 2202000335}


\begin{document}
\maketitle
\newpage
\subsection{Introducción}
\setstretch{1.15}
\bigskip
Es importante denotar la huella y la imagen de la mujer en cada uno de los ámbitos que la rodean, sobre todo si queremos entender la situación a la que se ven sometidas desde el inicio de la civilización como lo conocemos. 
\\\\
Es fácil intuir que su papel no ha sido de lo más dignificante, observando que aun hoy en día existen rezagos del papel de estas en algún punto de la historia. Como lo podemos observar en nuestro día a día, y a pesar de la constante lucha por la igualdad,  veremos que el papel de las mujeres se ve reducido en todos los aspectos de la vida pública, basta con alejarnos un poco de las grandes ciudades para hacer más visible las diferencias entre el papel de los hombres y las mujeres en las esferas políticas, sociales y laborales. 
\\\\
Para este caso en particular, abordaremos la perspectiva en un punto especifico en el tiempo, en concreto, la Edad Media, que desde mi muy personal punto de vista, es un época donde no existía un interés generalizado por ahondar en los temas de carácter social. 
Existía la esclavitud y la monarquía, dos puntos que contrastan muchísimo con la idea de igualdad para la sociedad, ya que no solo era poco alcanzable, si no que era prácticamente imposible salir de una esfera social para entrar a otra. 
\\\\
Dada situación no excluía a las mujeres, al contrario, las rodeaba con tal intensidad que es complicado de pensar que no fuera así. El mero hecho de que su opinión era tomada en cuenta era inimaginable, peor aún si las mujeres querían formar parte de las esferas sociales y políticas de la época. 
\\\\
Esto nos guiará a detallar la imagen de la mujer y como con base en la iconografía se puede dibujar de manera plena la posición de la misma en la sociedad, en el como interactuaba con sus iguales y como interactuaban con el sexo masculino, no hay mejor ejemplo para ello que las pinturas de la época, el arte en general, es una ventana bien abierta, por donde, de manera directa entran los retratos de una sociedad; el como pensaban, el como se comportaban y el como se relacionaban. 
\\\\
El estudio de estas imágenes nos permitirá vislumbrar no sólo la cultura del machismo que existía en la época, si no mejor aun, poder contrastarla con la situación actual, esto con el fin de demostrar la constante opresión, violencia y denigración hacia la mujer por parte de una sociedad, que en su mayoría, si no es que toda, machista. Pero igual notaremos si se hay algún punto en el que hayamos progresado con el paso de los años, solo con la intención de sentirnos consolados por el hecho de que vivimos en un mundo mejor. 
\\\\
Asimismo para complementar esta situación, se hará uso de ejemplos femeninos que se sobreponían a las condiciones y lograron ser la piedra angular de un movimiento que en el pasado no tiene el mismo significado que ahora; el feminismo. Este movimiento quizá no ve sus raíces en este momento, pero si lo dignifica y le da sentido al pasar de los años.  
\newpage
\subsection{El universo de la mujer: Espacio y Objetos. $\\$Françoise Piponnier}
\bigskip
En la Edad Media, las mujeres eran consideradas físicamente débiles y moralmente frágiles, y se creía que debían ser protegidas tanto de los demás como de los hombres mismos, una noción un tanto incongruente por parte de los hombres, ya que se supone que la mujer no tiene la capacidad de defenderse, pero incluso no tiene la capacidad de defendenderse del mismo hombre, porque el hombre asume que es el único que violenta a la mujer. 
\\\\
Dependiendo de su pertenencia por nacimiento o elección a diferentes esferas sociales, estaban bajo la vigilancia y dirección de los hombres de su respectiva comunidad, cosa que reafirma el punto pasado. Aquí llegamos a dos vertientes donde el mundo de la Iglesia era el único que confinaba completamente a las mujeres que normalmente eran religiosas abocadas a la iglesia, mientras que las mujeres en la clase de los guerreros tenían una vida más abierta y plena, aunque también estaban vigiladas. La situación de las mujeres en la clase de los trabajadores variaba y evolucionaba a lo largo de la Edad Media. Esto tiene un peso importante, considerando que aun hoy en día la situación es parecida, muchas mujeres conservan este estatus de vigilancia por parte de sus maridos. 
\\\\
Las mujeres tenían menos libertad de movimiento y acción que los hombres en todos los niveles de la sociedad. En cuanto al trabajo, las fuentes que nos pueden ayudar a entender son difusas y a menudo limitadas en información antes del siglo XIV. Pero sabemos que la  producción agrícola, en su mayoría, se consideraba un trabajo masculino, pero más tarde en algunas representaciones veremos a las mujeres involucradas en tareas agrícolas como la henificación y la cosecha, aunque las tareas que requerían un mayor esfuerzo estaban plenamente dedicadas al genero masculino. Esto nos deja pensando en las razones por las cuales se daba este suceso, podrían ser variadas las causas; una de ellas quizá la diferencia de fuerza física entre los dos géneros.
\\\\
En el ámbito de la artesanía, las mujeres participaban en su mayoría en los oficios, pero realmente solo se mencionaba a las mujeres solteras o viudas en los registros de oficios, que eran papeles que daban fé de las actividades de la sociedad. Las esposas de artesanos desempeñaban un papel importante en la comercialización de productos y podían participar en tareas de acabado o en pedidos importantes y urgentes. Esto tiene sentido si sabemos que las mujeres poseen una motricidad fina mayor a la del genero masculino, por lo cual podrían desarrollar tareas que requieren una mayor atención al detalle, pero al mismo tiempo nos hace pensar que justamente no sabemos porque razón no desarrollaban esta actividad con participación estelar, si no justamente a la sombra de un hombre. 
\\\\
Por otro lado, y reafirmando la idea pasada, la industria textil era un ámbito en el que las mujeres tenían un papel destacado, participando en todas las etapas, desde la selección y preparación de fibras hasta el tejido y el hilado. En algunos momentos y lugares, las mujeres también estaban involucradas en el trabajo de la seda y el lino. Esto quiere decir que no se le da el reconocimiento de manera completa a la mujer en todos los aspectos de la vida en el medievo. Uno podría asumir de manera sencilla que esto ocurre porque la industria textil no era precisamente la fuente económica principal en la época, contrario a la ganadería y la artesanía. Estamos hablando de una forma de ejercer dominio o poder sobre el genero femenino sobajando las actividades que realiza el mismo y poniendo como estandarte las realizadas por el hombre. 
\\\\
Además del trabajo, el principal papel de las mujeres en la sociedad medieval era el cuidado de la familia y la gestión de los bienes familiares. Esto implicaba tareas de crianza de niños, preparación de alimentos y gestión del hogar. Las mujeres también se encargaban de la recolección de leña y combustible, donde nuevamente caemos en un contexto que se repite hasta hoy en día. Muchos de los hogares en la actualidad tienen la figura de madre presente todo el día y conservando las mismas tareas. Se podría pensar en algún punto que esto está plenamente relacionado con la naturaleza paciente y más confortable de la mujer, y que en cierta manera las mujeres tienen un trato amable y empático, pero la realidad es que no tiene solo que ver con este aspecto, lo más seguro es que sea una delegación de labores que hombres consideraban sosos y sin interés o importancia. Hoy en día existe el termino "paternidad presente", donde se trata de compartir la educación de los hijos entre las dos figuras paternas y no solamente la figura femenina. 
\\\\
Algunas de las responsabilidades domésticas que eran claramente definidas por género, y muchas de ellas recaían en las mujeres se encuentra la de encender y mantener el fuego, la cual era una tarea exclusiva de las mujeres en el hogar, especialmente en las viviendas de tierra batida sin conducto para el humo. 
Las chimeneas eran un lujo reservado inicialmente para comunidades religiosas, castillos. Los lares eran exclusivos de campesinos ricos y habitantes urbanos acomodados. 
\\\\
En cuanto a la cocina, la cerámica era preferida al metal para guisos, aunque se usaban pequeñas cacerolas de bronce para la papilla de los niños. Además, se mencionan utensilios más especializados en inventarios burgueses, como herramientas para hacer tartas o asar carne.
\\\\
Dentro de este aspecto las mujeres también eran responsables de la gestión de agua, donde utilizaban distintas herramientas que varían según la región. La cerámica, como las ánforas, era esencial en las regiones mediterráneas y en regiones septentrionales como Borgoña, se usaban cántaros de agua más pequeños y cubos rodeados de hierro. 
\\\\
Resulta un poco curioso el analizar los dos párrafos pasados, ya que como bien mencionamos un poco antes, se podría pensar que los trabajos pesados era exclusivamente de hombres por el esfuerzo físico que estos podría requerir, pero la realidad es que la recolección de leña y agua tampoco es nada sencillo y sobre todo considerando que no existía como tal un medio de transporte que facilitará el trabajo, todo lo contrario, la persona era el medio de transporte en sí. Por lo que si podríamos concluir cierta discriminación hacia la mujer en estos casos, ya que entonces no se hace por una cuestión física. 
\\\\
En cuanto a la elaboración del pan, a pesar del desarrollo del oficio de panadero, muchas mujeres seguían amasando y cocinando pan en casa. Las mujeres también servían la comida en la mesa, y la iconografía muestra que en ocasiones solo los varones adultos se sentaban a la mesa, mientras que las mujeres servían y los hijos varones menores comían en un rincón. Esto queda mas que claro que es un aspecto en el que el hombre violenta no solo a la mujer si no a su familia entera, es notoria la necesidad del hombre de sobreponerse como la figura que tiene el poder, esto reflejado en el mero hecho del honor de comer en una mesa, que se reserva solamente al genero masculino, a la figura paterna, peor aun, se trata a la mujer como la servidumbre y se le enseña a servir, en mi perspectiva es una forma de educar a las generaciones futuras con este contexto de dominación sobre el genero femenino. 
\newpage
Respecto a la limpieza de la vajilla y de la casa era principalmente tarea de las mujeres, pero los detalles sobre las técnicas y los utensilios utilizados son escasos. Lo mismo ocurre con las actividades de limpieza corporal y de ropa. Estas son acciones que se repiten hasta la fecha, las labores del hogar están relacionadas ampliamente con el genero femenino, incluso en muchos hogares se les enseña de manera exclusiva a las niñas estas tareas, mientras que a los niños se les enseñan más labores relacionados con el trabajo o mantenimiento del hogar. 
\\\\
Aunque afortunadamente de un tiempo para acá la situación ha ido mejorando, ya que la mayoría de tareas del hogar se ven divididas entre los miembros de la familia y estoy seguro que poco a poco la situación terminara siendo equitativa para ambos géneros. Las tareas del hogar no deberían de ser una cuestión de genero, en lugar, deberían de ser percibidas como una parte de las reglas de comunidad de cualquier hogar. 
\subsubsection{El marco domestico}
Por otro lado, podríamos ahondar un poco más en como era el lugar en que las mujeres se desarrollaban. En la Edad Media, algunas mujeres trabajaban fuera de casa para contribuir a los gastos familiares, pero su principal responsabilidad era el cuidado de la familia, esto implicaba actividades realizadas principalmente dentro y cerca de la casa. Lo encontramos de gran interés ya que podemos asumir que la vida de las mujeres era restringido a las áreas cercanas al hogar. 
\\\\
El estudio de la arquitectura de las casas ha sido más común que el análisis de las actividades y la vida que tenían lugar en su interior. Algunos textos proporcionan descripciones detalladas de la disposición de las habitaciones, destacando la importancia de ciertos espacios para las mujeres.
\\\\
Se han identificado patrones en la disposición de las casas medievales, como la presencia de una habitación destinada a la preparación y consumo de alimentos, con utensilios de cocina y almacenamiento de agua. Estos espacios tenían una fuerte connotación femenina y también incluían objetos textiles y de adorno. Hablamos entonces de la fundamentación del hecho de que las mujeres tenían un amplio desarrollo en esto espacios del hogar más que en otro, incluso hablamos de una decoración con connotaciones femeninas que alude a un espacio meramente femenino donde no había ni la mínima intervención masculina, esto hoy en día no es tan común, pero sigue siendo tema de discusión en la agenda feminista, debido a que como bien se ha dicho, la alimentación o preparación de alimentos forma parte de lo que cualquier adulto funcional debería de realizar por si mismo y no es obligación de nadie realizarlo en lugar de la persona. 
\\\\
Como bien lo mencionamos las mujeres medievales a menudo salían de sus casas para realizar actividades cotidianas, como recolectar leña, obtener agua de fuentes o manantiales, o cuidar de huertos. Tales tareas las llevaban a lugares públicos donde se encontraban con otras mujeres y niños, esto formaba parte de las únicas actividades realizadas por mujeres en donde interactuaban con otras mujeres y no había la presencia de un hombre. En algunas casas urbanas, la presencia de hombres que trabajaban en oficios terciarios reducía la separación tradicional entre los espacios masculinos y femeninos en la casa, pero nuevamente forma parte de una minoría de los casos donde los roles de genero quedan muy establecidos y la misma sociedad patriarcal empuja a sus individuos a seguir participando de manera activa y ferviente en las actividades relacionadas con dichos roles. 
\newpage
\subsection{La mujer en las imagenes, la mujer imaginada $\\$Chiara Frugoni}
\bigskip
\subsubsection{El punto de vista de la Iglesia}
Este tema es de mi especial atención, ya que considero que la Iglesia ha participado activamente en la corriente de desprestigio hacia la mujer y de su minimizacion y sumisión ante el genero masculino. 
\\\\
En el Génesis, Eva fue maldecida con la carga de tener hijos, convirtiéndola en la culpable y marcando su destino como esposa y madre. Adán, en cambio, puede encontrar alegría en su trabajo, considerándolo un don de Dios y una forma de abrir la puerta al paraíso nuevamente. Esta afirmación extraída del análisis de la obra nos permite vislumbrar el significado de la mujer y el hombre dentro de la religión católica, en la que se puede observar la re afirmación de los absurdos roles, que además son considerados casi como castigo en lugar de una bendición, por la contra enaltecen actividades cotidianas realizadas por hombres, de nuevo notamos esta necesidad de dejar abajo de la figura masculina a la figura femenina. 
\\\\
La Iglesia occidental en la Edad Media hizo hincapié en la castidad como un ideal, y la virginidad se consideraba el estado más elevado, incluso se ponía a Eva en contraste con la Virgen María, quien aceptó su papel como intermediaria de la redención al llevar al hijo de Dios en su vientre. Las vírgenes se consideraban el estado más perfecto, mientras que los casados eran considerados como impuros. En pocas palabras la Iglesia aborrecía de fervientemente el sexo, pero con una cierta de sensación de que era odiado por involucrar a la mujer en el mismo y corromper este estado de máxima plenitud, logrando alejar a los hombres del estado de castidad. 
\\\\
La Iglesia occidental también tenía una visión negativa del matrimonio basada en las palabras de san Pablo, aunque san Agustín intentó tranquilizar a los fieles al respecto. La pareja fundadora de descendencia en el matrimonio no se consideraba un modelo. Comprobando la idea de que quizá la procreación o el acto del sexo sea ir en contra de este estado máximo al que todo ser humano debería de aspirar en su vida. La Iglesia tenía dificultades para evaluar positivamente el matrimonio, y había pocos santos casados.
\\\\
También mostraba incomodidad con José y María, retratando a José como un hombre mayor y a menudo representándolo dormido o de espaldas en las escenas de la Navidad. Esta posición reflejaba la iconografía antigua de la paternidad desconocida y permitía que el Espíritu Santo iluminara al niño divino de manera discreta. Esto me provoca una gran curiosidad, ya que antes hablábamos de la paternidad presente que prolifera hoy en día, es posible que en esta época tengamos como estándar el desentendimiento de los hijos gracias a la iglesia y este tipo de representaciones.
\\\\
La iconografía del matrimonio se percibía tan negativamente que a menudo se usaba en escenas de tentación diabólica, como en un capitel de la iglesia de la Madeleine en Vézelay del siglo XII, donde el diablo, con gestos similares al sacerdote en una boda, presenta a una mujer a san Benedicto. El "matrimonio" sugerido por el diablo era considerado infernal.
\\\\
Otro ejemplo similar se encuentra en un capitel de la iglesia de Civaux (Vienne) a principios del siglo XII, donde el matrimonio se representa como una rendición peligrosa a la tentación, con una pareja unida en la dextrarum iunctio junto a una sirena, símbolo de la lujuria y la seducción. Esto simbolizaba que aquellos que, incapaces de resistir las pasiones, se veían obligados a casarse estaban en peligro de perdición. San Bernardo, en un texto atribuido a él en el pasado, también comparaba a una mujer casada con una sirena.
\\\\
Los dos párrafos anteriores contienen un par de ideas que sería interesante desglosar: primero, la idea de que el matrimonio es la caída en la tentación diabólica, que bien podría ser por la idea que se tenía de la virginidad, pero que queda completamente derrumbada por la siguiente idea, que es la tentación y lujuria representada por un sirena, osease una figura femenina. Este hecho nos deja sobrentendido que el problema del matrimonio es con la mujer y no solo eso, si no con su naturaleza impura y diabólica. 
\\\\
A partir del siglo XII, la Iglesia, al proclamar el sacramento del matrimonio, regula esta institución con el objetivo de espiritualizarla. No solo prohíbe la violencia en la voluntad de los esposos y establece normas de consanguinidad, sino que también enfatiza la dignidad del contrato matrimonial, elevándolo a la categoría de sacramento. Aunque la doctrina de la virginidad y la castidad se mantiene, se produce una revalorización del matrimonio.
\\\\
Sin duda una de las estrategias más inteligentes por parte de la Iglesia, donde para solucionar estos problemas de perspectiva hacia la mujer, hacen del matrimonio una consagración para darle cierto nivel de espiritualidad. Osea, lo que no era un objetivo a alcanzar, lo forman parte de algo que se debe de realizar bajo la ley de Dios.
\subsubsection{En compañia del diablo}
En la representación artística y la literatura sobre la relación entre el diablo y los protagonistas, que se visten de santos, se encuentra el travestismo del diablo, adoptando la apariencia de una mujer para poner a prueba la virtud del protagonista. Un ejemplo de esto es un fresco en La Tebaide, atribuido a Buffalmacco en el Camposanto de Pisa, alrededor de 1343. En esta historia, el diablo se disfraza de una peregrina y trata de seducir a un monje ermitaño. El monje, sin embargo, reconoce el engaño y expulsa al diablo.
\\\\
Este tema se relaciona con la idea de que la presencia del demonio a menudo esta relacionado con las mujeres, lo que lleva a la creencia de que el demonio prefería el género femenino. La representación de mujeres poseídas era común en la literatura y el arte, y esto se interpretaba como la alienación y la insatisfacción que las mujeres experimentaban en la sociedad. El momento de la posesión diabólica simbolizaba la posibilidad de liberación, pero esta liberación era mala para la iglesia, estaba más relacionada con el demonio y se resolvía a través del exorcismo. En última instancia, este ciclo de posesión y exorcismo dejaba a las mujeres en un estado de olvido después de un breve momento de importancia como protagonistas.
\subsubsection{La condenada y su casa}
En la iglesia de San Lázaro en Autun del siglo XII, la misoginia es evidente en la representación del Juicio Universal en el tímpano. En este juicio, se destaca la desproporción numérica entre hombres y mujeres elegidos y condenados. Solo hay dos mujeres entre los elegidos, mientras que cuatro figuran entre los condenados. Esto justamente deja entrevisto que incluso dentro de las representaciones de la época hay una superposición de la imagen masculina sobre la femenina.
\\\\
En una representación artística en la iglesia de Santa Maria Maggiore en Toscana (Viterbo) del siglo XIV, se muestra a mujeres siendo devoradas por una cabeza de lobo con dientes afilados, mientras diablos las atormentan con tridentes. Estas mujeres, identificadas como monjas por sus velos blancos, muestran gestos de desesperación. Su ubicación en lo más profundo del Infierno refleja la gravedad de su traición a su voto de castidad y ascetismo. Es evidente que ni las mismas mujeres que pertenecen a la Iglesia tienen una posición favorable ante el trato por parte del genero masculino. 
\\\\
La presión ejercida por la Iglesia en la época promovía la misoginia tanto entre hombres como mujeres. A los hombres se les incitaba a desconfiar de las mujeres, incluso de las más santas, acusándolas de fingir y mentir. Mientras tanto, a las mujeres se les enseñaba a aceptar una imagen negativa de sí mismas, a pesar de que aquellas que renunciaban al matrimonio y elegían la penitencia estaban seguras de alcanzar el paraíso.
\subsubsection{El cuerpo seductor}
La misoginia en la Iglesia medieval se refleja en la representación de la mujer como la fuente de la tentación y el pecado, comenzando con Eva en el relato del pecado original. La serpiente tentadora, en la historia de Adán y Eva, es a menudo representada con un rostro de mujer, enfatizando la relación pecaminosa entre el hombre y la mujer. Las representaciones de mujeres seductoras, como sirenas, también contribuyen a esta imagen negativa. Algo de lo que se hablaba antes en el texto, incluso el uso de una figura femenina para representar criaturas míticas con connotación negativa. 
\\\\
El mensaje de la Iglesia promueve una profunda diferencia en el tratamiento de hombres y mujeres, considerando a los hombres como pecadores debido a sus acciones y a las mujeres como fuentes de tentación debido a su debilidad, incluso en algún punto se llega a considerar a la mujer incapaz de pecar, la mujer es sólo el medio para pecar. También la figura de la Muerte se representa como una vieja con rasgos de mujer, destacando la relación entre la mujer y la decadencia. En la literatura religiosa, las mujeres a menudo se despojan de su humanidad y se convierten en proyecciones del deseo masculino culpable.
\\\\
La vida de la Virgen María se presenta como un modelo alternativo, aunque su vida es considerada excepcional y fuera de las normas humanas. La Inmaculada Concepción y otros eventos relacionados con María subrayan su pureza y devoción, pero al mismo tiempo refuerzan los estereotipos misóginos al considerarla única e inimitable.
\subsubsection{La mujer como simbolo}
En el género literario de "La Batalla de los vicios y las virtudes" se representa la eterna lucha entre el bien y el mal. En un manuscrito del siglo XI, se ilustran parejas de vicios y virtudes contrapuestos. En la primera página, se muestra la avaricia representada por un hombre que está obsesionado con el dinero mientras aplasta a un campesino, mientras que la misericordia está representada por una figura que ayuda a un hombre desnudo y desamparado. En la segunda página, la lujuria es personificada por una mujer bellamente vestida y coronada por un diablo, mientras que la castidad está representada por una figura que vence al demonio.
\\\\
La elección de género de los personajes que representan los vicios y las virtudes a menudo se basa en estereotipos de género. La lujuria se representa como una mujer, ya que se considera que provoca el pecado en los hombres, mientras que la avaricia se representa como un hombre, ya que implica actividad y búsqueda de riqueza. Las mujeres a menudo se utilizan para representar conceptos abstractos, instituciones y virtudes, así como para simbolizar la lujuria.
\subsubsection{La mujer en la vida privada y cotidiana}
Hasta la Baja Edad Media, el arte principalmente tenía una orientación eclesiástica y reflejaba el punto de vista de la Iglesia. En parejas conyugales, el hombre solía ser quien ordenaba la representación, aunque en algunos casos, como el de Judith de Flandes en el siglo XI, una mujer, incluso de rango social elevado, se representaba junto a su esposo en una escena religiosa. Sin embargo, las representaciones de la Ecclesia en esta época empezaron a mostrar a la Virgen y a san Juan con atributos tradicionalmente masculinos, como un libro en la mano.
\\\\
En el contexto del matrimonio medieval, que a menudo se basaba en intereses y alianzas políticas, la mujer era considerada como un objeto en un intercambio entre el padre y el pretendiente. En algunas ocasiones, la representación de la pareja conyugal podía influir en la transmisión de la imagen de la mujer, especialmente si se esperaba un hijo importante. Esto se refleja en representaciones de parejas nobles como los padres de Ecberto, arzobispo de Tréveris, o Werner e Irmengarda, quienes se representan donando manuscritos a la iglesia, expresando su fidelidad al clero y a Dios. 
\\\\
Para ser claros, en el arte medieval, las mujeres no tenían una representación propia a menos que renunciaran al matrimonio y se consagraran a Dios, ya sea como viudas o solteras. Básicamente las mujeres no valían nada por si solas, solo tenían valor hasta que se les asociaba con un hombre y la Iglesia avalaba esta relación. 
\subsubsection{Encantar, curar}
La obsesión sobre la figura de la bruja en la Edad Media se relaciona con el temor de que las mujeres pudieran asumir prerrogativas masculinas y la preocupación sobre el poder de seducción que se asociaba a las mujeres. Esto se refleja en la persecución de brujas, que fueron quemadas en mayor cantidad que brujos.
\\\\
Un fresco en la iglesia de San Bernardino en Triora (Imperia) a finales del siglo XV muestra a un grupo de brujas en el infierno, con mitras en la cabeza y un diablo negro pintado sobre ellas como símbolo de su fe ciega. Se enfatiza en la biografía de la bruja su supuesta relación sexual con el demonio, la apariencia repugnante de vieja y su cruel trato hacia los niños recién nacidos, a quienes se les atribuía la muerte. Además, se les acusaba de fabricar ungüentos mágicos. Estos aspectos se relacionaban con las prácticas de las mujeres en la asistencia en partos y en la medicina popular para enfermedades femeninas. Lo cuál nos podrá resultar contradictorio, ya que se les asigna la tarea del cuidado del hogar y de la familia, pero luego se les acusa de practicar brujería por este hecho. 
\subsubsection{La mujer sabe leer}
En una miniatura del siglo XV, parte de la Historia escolástica de Juan de Ries, se representa una escena en una casa flamenca. Un anciano enfermo descansa en una especie de diván-caja de madera, mientras una criada se acerca con una escudilla de peltre para darle una poción. La esposa del enfermo, sentada junto a la chimenea, ha levantado su túnica para sentir el calor del fuego. Ella vierte una sustancia medicamentosa del caldero que cuelga sobre las llamas y sigue una receta escrita en un libro en su regazo. Esto destaca que la mujer, a pesar de ser de una clase acomodada, es una laica y es capaz de utilizar la palabra escrita. 
\\\\
Este ejemplo sugiere que el conocimiento de la lectura y la escritura podría haber estado más extendido de lo que generalmente se cree. Además, se mencionan otras representaciones que muestran a las mujeres en roles relacionados con la alfabetización de sus hijos, como una tabla del siglo XV donde la Virgen enseña el alfabeto a Jesús, o una escena en la que María, ya adolescente, aprende a leer en un salterio mientras su madre, Ana, le muestra las letras a deletrear.
\\\\
Christine de Pizan, una destacada escritora y copista, fue un ejemplo notable de una mujer que se mantuvo a sí misma y a su familia mediante su actividad literaria e iluminadora. Además, se destacan las representaciones de mujeres activas y creativas en obras como el "De claris mulieribus" de Boccaccio, que ofrece una variedad de modelos femeninos positivos, incluyendo artistas y creadoras. Sugiriendo que, a pesar de las restricciones de género de la época, las mujeres tenían la capacidad de participar en diversas formas de trabajo creativo y artístico, como la orfebrería y la escultura.
\subsubsection{Las esposas de Cristo}
La mayoría de las mujeres destacadas de la época eran monjas, como Roswitha, cronista e historiadora del siglo X, y Herrad, abadesa del siglo XII, autora de "Hortus deliciarum," una enciclopedia religiosa ricamente ilustrada.
\\\\
El cuarto del convento era el espacio donde las mujeres tenían "una habitación propia" para la oración, meditación, lectura y escritura. Muchas entraban en el convento siendo niñas y se dedicaban al estudio y a la copia de manuscritos. Las monjas podían ser comitentes autónomas y destinatarias de obras artísticas, como Hitda y Uta, que aparecen en miniaturas medievales ofreciendo manuscritos. Reafirmando así una idea que probablemente ya se había pensado, las mujeres solo pueden lograr cierto reconocimiento mediante la Iglesia, como mujeres que abandonan toda su vida para dedicarla a Cristo. 
\\\\
Aunque generalmente se asocia la copia de manuscritos a manos masculinas en monasterios, muchas generaciones de monjas silenciosas estuvieron dedicadas a copiar, iluminar y componer. Entre ellas, destaca Ende, una miniaturista del siglo X que firmó su trabajo en un manuscrito del Apocalipsis de Beato de Liébana. Otra figura notable es Guda, quien en el siglo XII escribió y pintó un homiliario, siendo uno de los primeros retratos de artista firmado y el más antiguo de una artista mujer conocido. Estas mujeres contribuyeron significativamente al mundo de la iluminación de manuscritos medievales y es aquí donde observamos que la mujer ha estado presente, sin retribución alguna. 
\subsubsection{Un tiempo para pensar}
La representación artística de la beata Umiltà en una tabla del siglo XIV ilustra la vida en un convento medieval, enfocándose en actividades relacionadas con la lectura, la enseñanza y la escritura. La tabla muestra a Umiltà, abadesa y fundadora de las hermanas vallombrosanas, leyendo en su celda, enseñando a las hermanas en el refectorio, desde el púlpito y dictando enseñanzas a dos de ellas que están escribiendo. Destaca la importancia de la educación y la erudición en la vida monástica. Es interesante notar que la comitente, en el centro de la escena, sostiene un libro, un atributo que anteriormente se asociaba principalmente a hombres.
\\\\
Catalina de Siena (1347-1380) es reconocida por su triple aureola, simbolizando su virginidad, su martirio y su papel como predicadora, a pesar de la prohibición bíblica de que las mujeres hablen en público. Catalina desafió esta restricción y fue reconocida como una auténtica predicadora y doctora de la Iglesia. Su habilidad para debatir y convencer se destaca en imágenes donde se la muestra enfrentando a filósofos paganos con libros en mano. Su elección de pertenecer a la orden tercera de los dominicos, que valoraba la cultura como herramienta de conversión, enfatiza su compromiso religioso y su capacidad intelectual.
\\\\
La difusión de la imagen de Catalina de Siena contribuyó a desafiar las normas de género de la época y a afirmar la idea de que las mujeres podían destacar en el campo del conocimiento y la erudición, no solo a través de la humildad y la obediencia, sino también a través del saber y la enseñanza.
\subsubsection{La santa patrona}
El auge de las ciudades medievales en Italia llevó al surgimiento de una religiosidad cívica, donde se canonizaban santos y santas locales, a menudo con cultos limitados a sus respectivas ciudades pero que encarnaban la identidad cívica y la devoción popular.
\\\\
Un ejemplo de esta devoción se encuentra en una tabla pintada por Niccolò Gerini en 1402, que representa a santa Fina en el centro con una representación en miniatura de San Gimignano. En los lados de la tabla se ilustran acontecimientos biográficos y milagros que muestran la participación activa de la santa en la vida de la ciudad, convirtiéndola en su patrona. Este fenómeno no era único y se observaba en otras ciudades italianas, donde la relación entre los habitantes y los edificios que constituían la ciudad se fortalecía.
\\\\
Este surgimiento de figuras femeninas como santas patronas refleja una lenta transformación en la posición de las mujeres en la sociedad medieval, donde comenzaron a desempeñar roles más activos y participativos en la vida cotidiana. Catalina de Siena, en particular, proclamó su devoción a la ciudad de Treviso, simbolizando el papel más destacado de las mujeres en la "época de los mercaderes". Esto marcó un cambio significativo en la percepción de las mujeres en la sociedad, tanto en la tierra como en el cielo.
\subsection{Conclusión}
Podemos observar a lo largo del texto la constante sobre posición de la figura del hombre en la de la mujer, como a lo largo de la historia, en concreto en el medievo la mujer y su palabra no tenían ningún valor, incluso al punto de ni siquiera ser candidatas de pensar por si mismo. Forman así parte de una sistema opresor que les dicta que hacer, que decir y como hacerlo, y en el momento en que se "salen del corral" se les acusa de lo peor, llegando a formar parte incluso de un asociación estúpida y tonta con el demonio. 
\\\\
Pero la realidad es que no era así, la importancia de las mujeres para la sociedad es más clara que nunca, no se podría fundamentar la sociedad sin la existencia de las mujeres, más allá del contexto de la preservación de la humanidad. Ellas forman parte de la piedra angular, ya que sus trabajos, que se repiten hoy en día, son de suma importancia. 
\end{document}          
